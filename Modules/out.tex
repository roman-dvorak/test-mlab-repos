\documentclass[11pt,a4paper]{article}
\usepackage[utf8]{inputenc}
\usepackage[english,czech]{babel}
\usepackage[T1]{fontenc}
\usepackage{amsmath}
\usepackage{amsfonts}
\usepackage{amssymb}
\usepackage{graphicx}
\usepackage[left=1.2cm,right=1.2cm,top=2cm,bottom=2cm]{geometry}
\usepackage[svgnames]{xcolor}
\usepackage{tabularx}
\usepackage{svg}
\usepackage{graphicx}
\usepackage{longtable}
\usepackage{booktabs}

%----------------------------------------------------------------------------------------
%	TITLE PAGE
%----------------------------------------------------------------------------------------

\author{Jan Chroust}
\title{SHT31V01A}
\begin{document}

\begin{tabularx}{\textwidth}{  l @{\extracolsep{\fill}} r }
\hline
\\
{\Huge SHT31V01A } & MLAB (logo)\\
\\
\hline
\end{tabularx}
%\maketitle
\vspace{2cm}
\begin{center}
{\Huge SHT31V01A – digitální vlhkoměr a teploměr}

{\Large Jan Chroust}
\end{center}

\vspace{2cm}




Jedná se o modul, který je možné osadit IO SHT30 nebo SHT31, které umí
měřit relativní vlhkost a teplotu s velkou přesností a stabilitou.
Rozsah měřené vlhkosti je 0 \% až 100 \%. Teplota je měřena v rozsahu
-40 C až 125 C. Komunikace probíhá přes rozhranní I2C.

\vspace{2cm}
\vfill

\section{Technické parametry}\label{technickuxe9-parametry}

\begin{longtable}[c]{@{}lll@{}}
\toprule
Parametr ~ ~ ~ ~ & Hodnota ~ ~ ~ & Poznámka ~ ~ ~ ~ ~ ~\tabularnewline
\midrule
\endhead
Relativní vlhkost & 0 \% - 100 \% ~ & Typ. přesnost dle
IO\tabularnewline
Teplota ~ ~ ~ ~ ~ & -40C - 125C ~ & Typ. přesnost dle IO\tabularnewline
integrovaný obvod & SHT30, SHT31 & ~ ~ ~ ~ ~ ~ ~ ~ ~ ~\tabularnewline
\bottomrule
\end{longtable}


\clearpage
\newpage
\pagestyle{empty} % Removes page numbers

\section{Popis konstrukce}\label{popis-konstrukce}

Jedná se o modul založený na IO SHT31V01A, který umožňuje měření
relativní vlhkosti a teploty a velkou přesností a stabilitou. Další
přesné informace IO je možné vyčíst z oficiálního dokumentačního listu
výrobce. Modul obsahuje veškeré potřebné součástky pro správný chod.

\section{Osazení a oživení}\label{osazenuxed-a-oux17eivenuxed}

\subsection{Osazení}\label{osazenuxed}

\subsection{Oživení}\label{oux17eivenuxed}

Je potřeba provést kontrolu zda není na plošném spoji zkrat a zda je
dobře zapájen IO. Jinak není třeba nic oživovat, pouze připojit a napsat
program. Když je nulovým odporem osazena pozice R4 adresa modulu je
0x44, pokud je osazena pozice R3 je adresa 0x45.

\subsection{Program}\label{program}

Vzorový program se nachází ve složce SW modulu. Pro spuštění je potřeba
mít nainstalovaný pyMLAB.


\end{document}